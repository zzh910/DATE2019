\section{Experimental Result}
\label{sec:result}
Data set is obtained from ISPD'08 benchmarks because NTHU-Route 2.0 used in this work only accepts this format. The experiment was run on linux workstation with X-core X.XXGHz CPU and XXGB memory. Since the router uses L-shape to calculate the congestion map, the 2-pin usage attribute is set to consider L-shape only, along with which other attributes are also extracted for training models and estimating congestion. The difference between the routing performance with and without our model can be considered the impact of hidden interactions captured by our prediction model.

\subsection{Runtime comparison}
The NTHU-Route 2.0 uses bounding-box routing technique to generate the initial congestion map on which the routability-driven edge shifting is performed. However, for the purpose of routing and other applications such as routability-driven placement, the final congestion map produced by NTHU is generated after the edge shifting and L-shape pattern routing. The runtime comparison is shown in Table \ref{tab:runtime}. The runtime of the bounding box routing algorithm, full estimation and extraction are all directly proportional to the design size. However, extraction is much simpler in contrast to the routing algorithms. We achieved speedups of 9.33X and 17.45X, comparing with bounding box routing and NTHU-Route's estimation method.
\begin{table*}[htbp]
\caption{congestion estimation runtime comparison}
\begin{center}
\begin{tabular}{|c|c|c|c|c|c|}
\hline
\multirow{2}{*}{Benchmark} & \multicolumn{2}{c|}{NTHU-Route 2.0 \cite{NTHU}} & \multicolumn{3}{c|}{Our Work}        \\ \cline{2-6} 
                           & Bbox Route(sec)   & Full Estimation(sec)  & Attribute Extraction(sec) & Prediction(sec) & Total(sec) \\ \hline
adaptec1                   & 2.02         & 6.49             & 0.42                 & 0.11       & 0.53  \\ \hline
adaptec2                   & 2.74         & 5.81             & 0.35                 & 0.13       & 0.48  \\ \hline
adaptec3                   & 11.25        & 20.36            & 0.63                 & 0.5        & 1.13  \\ \hline
adaptec4                   & 12.16        & 20.62            & 0.58                 & 0.51       & 1.09  \\ \hline
adaptec5                   & 9.52         & 19.08            & 0.89                 & 0.22       & 1.11  \\ \hline
bigblue1                   & 1.60         & 4.93             & 0.31                 & 0.05       & 0.36  \\ \hline
newblue1                   & 2.29         & 4.74             & 0.39                 & 0.2        & 0.59  \\ \hline
newblue2                   & 5.22         & 9.85             & 0.47                 & 0.20       & 0.67  \\ \hline
newblue5                   & 19.71        & 34.5             & 1.18                 & 0.35       & 1.53  \\ \hline
newblue6                   & 13.90        & 24.06            & 0.96                 & 0.17       & 1.13  \\ \hline
Total Ratio                      & 9.33        & 17.45             & n/a                  & n/a        & 1.00 \\ \hline
\end{tabular}
\label{tab:runtime}
\end{center}
\end{table*}

\subsection{Impact on routability-driven edge shifting}
In order to evaluate the quality of Steiner Tree structure, after all nets of a design has done routability-driven edge shifting, we perform L-shape routing on the optimised topology. The bounding box routing estimation only considers the boundaries of 2-pin boxes, which is the equivalent of performing L-shape routing twice without taken into account other factors. As is mentioned in \cite{fastroute}, only the same technique of routing is being applied for congestion estimation, can we obtain an relatively accurate congestion map for that router. Therefore, L-shape routing is chosen for routing.

We first take adaptec3 as an special case and exclude it, since its total overflow is too abnormal while other two factors stayed in range. This might happen due to the uncontrollable model behaviour which we have not yet fully investigated. As is shown in Table \ref{tab:treequality}, maximum overflow (\textit{Max\_OF}) has decreased 18\% and total overflow (\textit{Total\_OF}) has dropped 9\%. Please note that the wirelength (\textit{WL}) has only changed less than 1\%, which means almost no detour is made, the Steiner Tree topology optimised using our congestion map did avoid congested regions before routing is carried out.



\begin{table*}[htbp]
\caption{Result of Steiner Tree quality after edge shifting}
\begin{center}
\begin{tabular}{|c|c|c|c|c|c|c|}
\hline
\multirow{2}{*}{Benchmark} & \multicolumn{3}{c|}{Bounding Box Routing}     & \multicolumn{3}{c|}{Our Work}  \\ \cline{2-7} 
                           & WL       & Max\_OF & Total\_OF      & WL       & Max\_OF & Total\_OF \\ \hline
adaptec1                   & 3391427  & 112     & 245459 (+20\%) & 3391361  & 77      & 203315    \\ \hline
adaptec2                   & 3207461  & 93      & 152268 (+15\%) & 3207459  & 92      & 132709    \\ \hline
adaptec3                   & 9346466  & 80      & 593070 (-90\%) & 9346462  & 75      & 4947842   \\ \hline
adaptec4                   & 8891886  & 75      & 220259 (+63\%) & 8891797  & 63      & 134954    \\ \hline
adaptec5                   & 9505500  & 132     & 603652 (+8\%)  & 9805443  & 116     & 559175    \\ \hline
bigblue1                   & 3433454  & 76      & 218667 (+1\%)  & 3433406  & 66      & 217006    \\ \hline
newblue1                   & 2325703  & 52      & 52556 (-1\%)   & 2325657  & 43      & 52821     \\ \hline
newblue2                   & 4604357  & 89      & 77374 (+42\%)  & 4604356  & 67      & 54487     \\ \hline
newblue5                   & 14230198 & 155     & 627669 (+13\%) & 14230009 & 123     & 556639    \\ \hline
newblue6                   & 9755863  & 103     & 475561 (-12\%) & 9755823  & 106     & 544858    \\ \hline
Total Ratio (ex. adaptec3)       & 0.99     & 1.18    & 1.09           & 1        & 1       & 1         \\ \hline
\end{tabular}
\label{tab:treequality}
\end{center}
\end{table*}

\subsection{Global Routing Performance}
The full routing process consists of two parts: Routing and post-processing. The post-processing is a process to erase the history cost of all edges from the first stage, so that the edges previously repeatedly visited but not yet heavily occupied can be freed as valid candidates again. The executable compiled from NTHU-Route's unmodified source code could not successfully run through the post-processing for some of the designs. Therefore, we are only comparing the outputs of the first phase. The overflow threshold is set to 200, which means the router will proceed to post-processing when overflow goes below 200 during iterations.

We can observe from the Table \ref{tab:gr} that by applying our congestion model, the \textit{Total\_OF} of all designs are smaller than corresponding ones, summed up in a total of 10\% improvement. Lower \textit{Total\_OF} represents a higher chance of being successfully routed. Moreover, it would lighten the task of the post-processing. Although the total runtime of both are close, we achieved an average of 4.8\% acceleration. Also worth noticing, the \textit{Total\_OF} of adaptec2, adaptec3, and newblue2 all have a large quantity gap between using the original and our congestion map. by modifying the threshold value based upon these experimental data, we can reduce the number of iterations. Other factors are tied, which means it is feasible that we can replace the original global-routing based congestion estimation by our supervised-learning based congestion prediction model.


\begin{table*}[htbp]
\caption{Result of Global Routing Performance}
\begin{center}
\begin{tabular}{|c|c|c|c|c|c|c|c|c|c|c|}
\hline
\multirow{2}{*}{Benchmark} & \multicolumn{5}{c|}{NTHU-Route 2.0}                           & \multicolumn{5}{c|}{NTHU-Route 2.0 with Our Work}      \\ \cline{2-11} 
                           & \#iteration & Max\_OF & Total\_OF & WL       & Time(s)        & \#iteration & Max\_OF & Total\_OF & WL       & Time(s) \\ \hline
adaptec1                   & 11          & 2       & 170       & 3595770  & 651.60 (+1\%)  & 11          & 2       & 166       & 3593832  & 647.43  \\ \hline
adaptec2                   & 12          & 2       & 197       & 3307228  & 224.24 (0\%)   & 12          & 2       & 175       & 3306489  & 223.94  \\ \hline
adaptec3                   & 8           & 2       & 177       & 9471843  & 608.27 (+8\%)  & 8           & 2       & 128       & 9670341  & 561.43  \\ \hline
adaptec4                   & 4           & 4       & 116       & 8970266  & 119.74 (+36\%) & 4           & 4       & 115       & 8967345  & 77.40   \\ \hline
adaptec5                   & 14          & 2       & 173       & 10307879 & 1844.65 (+4\%) & 14          & 2       & 143       & 10306712 & 1774.39 \\ \hline
bigblue1                   & 16          & 2       & 143       & 3719329  & 1407.23 (-4\%) & 15          & 2       & 198       & 3716884  & 1473.37 \\ \hline
newblue1                   & 23          & 2       & 191       & 2398649  & 956.99 (-9\%)  & 23          & 2       & 184       & 2402800  & 1051.33 \\ \hline
newblue2                   & 3           & 4       & 197       & 4643206  & 52.90 (+12\%)  & 4           & 4       & 112       & 4642987  & 43.53   \\ \hline
newblue5                   & 16          & 2       & 182       & 14701827 & 1924.45 (0\%)  & 16          & 2       & 174       & 14701360 & 1925.38 \\ \hline
newblue6                   & 17          & 2       & 172       & 10270966 & 3229.39 (0\%)  & 17          & 2       & 172       & 10270966 & 3224.76 \\ \hline
Total Ratio                & 1           & 1       & 1.10      & $\approx$1     & $\approx$1           & 1           & 1       & 1         & 1        & 1       \\ \hline
Avg. \%change              &             &         &           &          & +4.8\%         &             &         &           &          &         \\ \hline
\end{tabular}
\label{tab:gr}
\end{center}
\end{table*}