\begin{table*}[htbp]
    \centering
    \caption{congestion estimation runtime comparison}
    \label{tab:runtime}
    %{{{
    \begin{tabular}{c|c|c|c|c|c}
        \toprule
        \multirow{2}{*}{Benchmark} & \multicolumn{2}{c|}{NTHU-Route 2.0 \cite{NTHU}} & \multicolumn{3}{c}{Our Work}        \\ \cline{2-6} 
                           & Bbox Route(sec)   & Full Estimation(sec)  & feature Extraction(sec) & Prediction(sec) & Total(sec) \\ \hline
        adaptec1           & 2.02         & 6.49             & 0.42                 & 0.11       & 0.53  \\ \hline
        adaptec2           & 2.74         & 5.81             & 0.35                 & 0.13       & 0.48  \\ \hline
        adaptec3           & 11.25        & 20.36            & 0.63                 & 0.5        & 1.13  \\ \hline
        adaptec4           & 12.16        & 20.62            & 0.58                 & 0.51       & 1.09  \\ \hline
        adaptec5           & 9.52         & 19.08            & 0.89                 & 0.22       & 1.11  \\ \hline
        bigblue1           & 1.60         & 4.93             & 0.31                 & 0.05       & 0.36  \\ \hline
        newblue1           & 2.29         & 4.74             & 0.39                 & 0.2        & 0.59  \\ \hline
        newblue2           & 5.22         & 9.85             & 0.47                 & 0.20       & 0.67  \\ \hline
        newblue5           & 19.71        & 34.5             & 1.18                 & 0.35       & 1.53  \\ \hline
        newblue6           & 13.90        & 24.06            & 0.96                 & 0.17       & 1.13  \\ \hline
        Total Ratio        & 9.33         & 17.45            & n/a                  & n/a        & 1.00 \\ \bottomrule
    \end{tabular}
    %}}}
\end{table*}

\begin{table*}[htbp]
    \centering
    \caption{Result of Steiner Tree quality after edge shifting}
    \label{tab:treequality}
    %{{{
    \begin{tabular}{|c|c|c|c|c|c|c|c|}
\hline
\multirow{2}{*}{Benchmark} & \multicolumn{4}{c|}{Bounding Box Routing}           & \multicolumn{3}{c|}{Our Work}  \\ \cline{2-8} 
                           & WL       & Max\_OF & Total\_OF & Total\_OF \%Change & WL       & Max\_OF & Total\_OF \\ \hline
adaptec1                   & 3391427  & 112     & 245459    & +20\%              & 3391361  & 77      & 203315    \\ \hline
adaptec2                   & 3207461  & 93      & 152268    & +15\%              & 3207459  & 92      & 132709    \\ \hline
adaptec3                   & 9346466  & 80      & 593070    & +20\%              & 9346462  & 75      & 494784    \\ \hline
adaptec4                   & 8891886  & 75      & 220259    & +63\%              & 8891797  & 63      & 134954    \\ \hline
adaptec5                   & 9505500  & 132     & 603652    & +8\%               & 9805443  & 116     & 559175    \\ \hline
bigblue1                   & 3433454  & 76      & 218667    & +1\%               & 3433406  & 66      & 217006    \\ \hline
newblue1                   & 2325703  & 52      & 52556     & -1\%               & 2325657  & 43      & 52821     \\ \hline
newblue2                   & 4604357  & 89      & 77374     & +42\%              & 4604356  & 67      & 54487     \\ \hline
newblue5                   & 14230198 & 155     & 627669    & +13\%              & 14230009 & 123     & 556639    \\ \hline
newblue6                   & 9755863  & 103     & 475561    & -12\%              & 9755823  & 106     & 544858    \\ \hline
Total Ratio                & 1.00     & 1.18    & 1.11      & +16.9\%            & 1.00     & 1.00    & 1.00      \\ \hline
\end{tabular}
    %}}}
\end{table*}

\section{Experimental Result}
\label{sec:result}
Data set is obtained from ISPD'08 benchmarks because NTHU-Route 2.0 used in this work only accepts this format. The experiment was run on the linux workstation with 4-core 2.6GHz CPU and 8GB memory. Since the router uses L-shape to calculate the congestion map, the 2-pin usage feature is set to consider L-shape only, along with other features  extracted for training models and estimating congestion. The difference between the routing performance with and without our model can be considered the impact of hidden interactions captured by our prediction model.

\subsection{Runtime Comparison}
The NTHU-Route 2.0 uses bounding-box routing technique to generate the initial congestion map on which the routability-driven edge shifting is performed. However, for the purpose of routing and other applications such as routability-driven placement, the final congestion map produced by NTHU is generated after the edge shifting and L-shape pattern routing. The runtime comparison is shown in \Cref{tab:runtime}. The runtime of the bounding box routing algorithm, full estimation and extraction are all directly proportional to the design size. However, extraction is much simpler in contrast to the routing algorithms. We achieved speedups of 9.33X and 17.45X, comparing with bounding box routing and NTHU-Route's estimation method.

\subsection{Impact on Routability-Driven Edge Shifting}
In order to evaluate the quality of Steiner Tree structure, after all nets of a design has done routability-driven edge shifting, we perform L-shape routing on the optimised topology. The bounding box routing estimation only considers the boundaries of 2-pin boxes, which is the equivalent of performing L-shape routing twice without taken into account other factors. As is mentioned in \cite{fastroute}, only the same technique of routing is being applied for congestion estimation, can we obtain an relatively accurate congestion map for that router. Therefore, L-shape routing is chosen for routing.

As is shown in \Cref{tab:treequality}, maximum overflow (\text{Max\_OF}) has decreased 18\% and total overflow (\text{Total\_OF}) has dropped 11\%. Please note that the wirelength (\text{WL}) has barely changed, which means almost no detour is made, the routing topology optimised using our congestion model did avoid congested regions before routing is carried out.


\begin{table*}[htbp]
    \centering
    \caption{Result of Global Routing Performance}
    \label{tab:gr}
    %{{{
    \begin{tabular}{c|c|c|c|c|c|c|c|c|c|c|c}
        \toprule
        \multirow{2}{*}{Benchmark} & \multicolumn{6}{c|}{NTHU-Route 2.0}                                     & \multicolumn{5}{c}{NTHU-Route 2.0 with Our Work}      \\ \cline{2-12} 
                           & \#Iteration & Max\_OF & Total\_OF & WL       & Time(s) & Time \%Change & \#Iteration & Max\_OF & Total\_OF & WL       & Time(s) \\ \hline
        adaptec1                   & 11          & 2       & 170       & 3595770  & 651.60  & +1\%           & 11          & 2       & 166       & 3593832  & 647.43  \\ \hline
        adaptec2                   & 12          & 2       & 197       & 3307228  & 224.24  & 0\%            & 12          & 2       & 175       & 3306489  & 223.94  \\ \hline
        adaptec3                   & 8           & 2       & 177       & 9471843  & 608.27  & +8\%           & 8           & 2       & 128       & 9670341  & 561.43  \\ \hline
        adaptec4                   & 4           & 4       & 116       & 8970266  & 119.74  & +36\%          & 4           & 4       & 115       & 8967345  & 77.40   \\ \hline
        adaptec5                   & 14          & 2       & 173       & 10307879 & 1844.65 & +4\%           & 14          & 2       & 143       & 10306712 & 1774.39 \\ \hline
        bigblue1                   & 16          & 2       & 143       & 3719329  & 1407.23 & -4\%           & 15          & 2       & 198       & 3716884  & 1473.37 \\ \hline
        newblue1                   & 23          & 2       & 191       & 2398649  & 956.99  & -9\%           & 23          & 2       & 184       & 2402800  & 1051.33 \\ \hline
        newblue2                   & 3           & 4       & 197       & 4643206  & 52.90   & +12\%          & 4           & 4       & 112       & 4642987  & 43.53   \\ \hline
        newblue5                   & 16          & 2       & 182       & 14701827 & 1924.45 & 0\%            & 16          & 2       & 174       & 14701360 & 1925.38 \\ \hline
        newblue6                   & 17          & 2       & 172       & 10270966 & 3229.39 & 0\%            & 17          & 2       & 172       & 10270966 & 3224.76 \\ \hline
        Total Ratio                & 1.00        & 1.00    & 1.10      & 1.00     & 1.00    & +4.8\%         & 1.00        & 1.00    & 1.00      & 1.00     & 1.00    \\ \bottomrule
    \end{tabular}
    %}}}
\end{table*}


\subsection{Global Routing Performance}
The full routing process consists of two parts: routing and post-processing. The post-processing is a process to erase the history cost of all edges from the first stage, so that the edges repeatedly visited but not selected due to high congestion can be freed as valid candidates again. This process resets the impact brought by our congestion model to the router. Therefore, we are comparing the outputs of the first routing phase to illustrate the effect of our congestion model on routing. The overflow threshold is set to 200, which means the router will proceed to post-processing when overflow goes below 200 during iterations.

We can observe from the \Cref{tab:gr} that by applying our congestion model, the \text{Total\_OF} of all designs are smaller than corresponding ones, summed up in a total of 10\% improvement. Lower \text{Total\_OF} represents a higher chance of being successfully routed. Moreover, we achieved an average of 4.8\% acceleration in computational time. Also worth noticing, the \text{Total\_OF} of adaptec2, adaptec3, and newblue2 all have a large quantity gap between using the original and our congestion map. by modifying the threshold value based upon these experimental data, we can reduce the number of iterations. Other factors are tied, which means it is feasible that we can replace the original global-routing based congestion estimation by our supervised-learning based congestion prediction model.


